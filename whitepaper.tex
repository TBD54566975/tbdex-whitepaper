% Options for packages loaded elsewhere
\PassOptionsToPackage{unicode}{hyperref}
\PassOptionsToPackage{hyphens}{url}
%
\documentclass[
]{article}
\usepackage{amsmath,amssymb}
\usepackage{iftex}
\ifPDFTeX
  \usepackage[T1]{fontenc}
  \usepackage[utf8]{inputenc}
  \usepackage{textcomp} % provide euro and other symbols
\else % if luatex or xetex
  \usepackage{unicode-math} % this also loads fontspec
  \defaultfontfeatures{Scale=MatchLowercase}
  \defaultfontfeatures[\rmfamily]{Ligatures=TeX,Scale=1}
\fi
\usepackage{lmodern}
\ifPDFTeX\else
  % xetex/luatex font selection
\fi
% Use upquote if available, for straight quotes in verbatim environments
\IfFileExists{upquote.sty}{\usepackage{upquote}}{}
\IfFileExists{microtype.sty}{% use microtype if available
  \usepackage[]{microtype}
  \UseMicrotypeSet[protrusion]{basicmath} % disable protrusion for tt fonts
}{}
\makeatletter
\@ifundefined{KOMAClassName}{% if non-KOMA class
  \IfFileExists{parskip.sty}{%
    \usepackage{parskip}
  }{% else
    \setlength{\parindent}{0pt}
    \setlength{\parskip}{6pt plus 2pt minus 1pt}}
}{% if KOMA class
  \KOMAoptions{parskip=half}}
\makeatother
\usepackage{xcolor}
\setlength{\emergencystretch}{3em} % prevent overfull lines
\providecommand{\tightlist}{%
  \setlength{\itemsep}{0pt}\setlength{\parskip}{0pt}}
\setcounter{secnumdepth}{-\maxdimen} % remove section numbering
\ifLuaTeX
  \usepackage{selnolig}  % disable illegal ligatures
\fi
\IfFileExists{bookmark.sty}{\usepackage{bookmark}}{\usepackage{hyperref}}
\IfFileExists{xurl.sty}{\usepackage{xurl}}{} % add URL line breaks if available
\urlstyle{same}
\hypersetup{
  hidelinks,
  pdfcreator={LaTeX via pandoc}}

\author{}
\date{}

\begin{document}

\textbf{tbDEX : A Liquidity Protocol v0.2}

@TBD54566975

\textbf{Abstract}. tbDEX is a protocol for discovering liquidity and
exchanging assets (such as fiat money, real world goods, stablecoins or
bitcoin)when the existence of social trust is an intractable element of
managing transaction risk. The tbDEX protocol facilitates decentralized
networks of exchange between assets by providing a framework for
establishing social trust, utilizing decentralized identity (DID) and
verifiable credentials (VCs) to establish the provenance of identity in
the real world. The protocol has no opinion on anonymity as a feature or
consequence of transactions. Instead, it allows willing counterparties
to negotiate and establish the minimum information acceptable for the
exchange. Moreover, it provides the infrastructure necessary to create a
ubiquity of on-ramps and off-ramps directly between the fiat and
decentralized financial systems without the need for centralized
intermediaries and trust brokers. This makes currencies and
decentralized financial services more accessible to everyone.

1 Introduction

We are at a crossroads in our financial system. The emergence of
trustless, decentralized networks unlocks the potential for a future
where commerce can happen without the permission, participation, or
benefit of financial intermediaries.

Globally, 1.7 billion adults lack access to the banking system, yet
two-thirds of them own a mobile phone that could help them access
financial services {[}1{]}. The reasons for their exclusion vary, but
the common threads are cost, risk, and lack of infrastructure.
Decentralized and permissionless systems create a world that empowers
individuals --- one in which the right to engage in payments is neither
subject to proving creditworthiness and the ability to pay account fees,
nor subject to censorship when an intermediary's values do not comport
with the payer or payee. It's also a world where internet access is the
only fundamental infrastructure required to participate.

An open, decentralized financial system will enable all people to
exchange value and transact with each other globally, securely, and at
significantly lower cost and more inclusively than what traditional
financial systems allow.

tbDEX was formed out of a desire to enable everyone to realize this
vision of the future. The current state decentralized financial systems
is still beyond the reach of everyday people, mostly requiring multiple
asset transfers and transaction fees each step of the way. Aside from
gatekeepers and cost, the complexity and sheer unintelligibility of this
process today is a prohibitive barrier to entry for most. The tbDEX
protocol is directed at this problem.

The protocol provides a framework for creating on-ramps and off-ramps
from systems of fiat to digital currencies, without the need for going
through centralized exchanges. The protocol affords for the secure
exchange of identity and mechanisms for allowing participants to comply
with laws and regulations (a feature notably absent from many efforts in
the decentralized finance world).

At its core, the tbDEX protocol facilitates the formation of networks of
mutual trust between counterparties that are not centrally controlled;
it allows participants to negotiate trust directly with each other (or
rely on mutually trusted third-parties to vouch for counterparties), and
price their exchanges to account for perceived risk and specific
requirements.

2. Foundational Concepts

\textbf{Trust}

The tbDEX protocol approaches trust differently than other decentralized
exchange protocols in the sense that it does not utilize a trustless
model, such as atomic swaps. At first blush, this is not optimal,
especially when considering the end goal of providing access to a
trustless asset like bitcoin. However, the reality is that no interface
with the fiat monetary system can be trustless; the endpoints on fiat
rails will always be subject to regulation, and there will exist the
potential for bad behavior on the part of counterparties. This means
that any exchange of value must be fundamentally based on other means of
governing trust --- particularly reputation.

The tbDEX protocol borrows heavily, if not completely, from
well-established models of decentralizing trust, such as the public key
infrastructure (PKI) that is used for securing the internet today.

Building on top of Decentralized Identifiers (DID) {[}2{]}, this
specification lays out a trust model in which trust is governed through
disparate verifiers of trust; this is ultimately in the control of
individuals, implementers of digital currency wallets, and/or delegates
of trust established by either group.

The protocol itself does not rely on a federation to control permission
or access to the network. There is no governance token. In its most
abstract form, it is an extensible messaging protocol with the ability
to form distributed trust relationships as a core design facet. The
protocol itself has no opinion on what an optimal trust relationship
between an individual wallet and a participating financial institution
(PFI) should look like.

The nature of this trust relationship will never be universal: different
jurisdictions are subject to different laws and regulations; and
different individuals and institutions will have varying levels of risk
tolerance, influenced by price and other incentives. It would violate
the principle of trying to achieve the maximum amount of
decentralization if the negotiation of trust was dictated at the
protocol layer, as that would necessarily involve some form of
permissioned federation.

2.1 Decentralized Identifiers (DIDs)

Decentralized identifiers (DIDs) {[}2{]} are a new type of identifier
that enables verifiable, decentralized digital identity. A DID refers to
any subject (e.g., a person, organization, thing, data model, abstract
entity, etc.) determined by the controller of the DID. In contrast to
typical federated identifiers, DIDs have been designed so they may be
decoupled from centralized registries, identity providers, and
certificate authorities. Specifically, while other parties may be used
to help enable the discovery of information related to a DID, the design
enables the owner of a DID to prove control over it without requiring
permission from any other

party. DIDs are Uniform Resource Identifiers (URIs) that associate a DID
subject with a DID document, allowing trustworthy interactions
associated with that subject.

DIDs are linked to DID Documents, a metadata file that contains two
primary data elements:

\begin{quote}
1. Cryptographic material the DID owner can use to prove control over
the associated DID (i.e. public keys and digital signatures)
\end{quote}

\begin{quote}
2. Routing endpoints for locations where one may be able to contact or
exchange data with the DID owner (e.g. location where PFI can be
accessed)
\end{quote}

DID Methods may be implemented in very different ways, but the following
are essential attributes of exemplar Methods:

\begin{itemize}
\tightlist
\item
  The system must be open, public, and permissionless.
\item
  The system must be robustly censorship resistant and tamper evasive.
\item
  The system must produce a record that is probabilistically finalized
  and independently, deterministically verifiable, even in the presence
  of segmentation, state withholding, and collusive node conditions.
\item
  The system must not be reliant on authorities, trusted third-parties,
  or entities that cannot be displaced through competitive market
  processes.
\end{itemize}

2.2 Verifiable Credentials (VCs)

Credentials are a part of our daily lives: driver's licenses are used to
assert that we are capable of operating a vehicle; and diplomas are used
to indicate the completion of degrees. In the realm of business, there
exist signed receipts for payments, consumer reviews of products, and
countless assertions made between individuals and non-governmental
parties. While all these credentials provide benefits to us within apps,
platform silos, and isolated interactions, there exists no uniform,
standardized means to convey generalized digital credentials that are
universally verifiable across domains, federation boundaries, and the
Web at large.

The Verifiable Credentials specification provides a standard way to
express credentials across the digital world in a way that is
cryptographically secure, privacy respecting, and machine verifiable.
The addition of zero-knowledge proof (ZKProof) {[}3{]} cryptography to
VC constructions (e.g. SNARK credentials) {[}4{]} can further advance
privacy and safety by preventing linkability across disclosures,
reducing the amount of data disclosed, and in some cases removing the
need to expose raw data values at all where legal and compliance needs
do not require it.

3 Participants

3.1 Issuers of Verifiable Credentials

Issuers are the source of VCs. Both organizations and individuals (by
means of their wallet) can be an Issuer. For example, a reputable
organization that already conducts KYC checks could begin issuing a KYC
credential to individuals. A wallet could also issue an evaluation of a
PFI that it had a negative experience with and circulate this amongst
their network, effectively acting as verifiable reputational feedback.

An incentive that may appeal to an Issuer is the potential to charge a
PFI for the issuance of a VC used to provide a sense of credibility or
legitimacy downstream. It's worth noting that verifiers, which can be a
PFI, a wallet, or an individual do not have to establish an explicit or
direct relationship with an Issuer in order to receive or verify
credentials issued by them. Instead, a verifier need only decide whether
they are willing to make a business decision based on the level of trust
assurance they have in the issuer of a given credential.

3.2 Wallets

Wallets act as agents for individuals or institutions by facilitating
exchanges with PFIs. More specifically, a wallet provides, though is not
limited to, the following functionalities:

• Providing secure encrypted storage for VCs\\
• PFI discovery\\
• Receiving, offering, and presenting VCs\\
-- Note: end user consent would be required to offer VCs\\
• Applying digital signatures\\
• Storing transaction history

Wallets developed using the tbDEX protocol significantly simplify the
user experience for their customers seeking to move assets between fiat
and digital currencies. Individuals or organizations would no longer be
required to first onboard through a separate, centralized exchange to
procure digital currency assets with fiat payment instruments, before
transferring those digital currency assets into the wallets. Individuals
or organizations can also leverage the protocol to easily off-ramp back
into fiat.

The protocol enables wallets to provide a streamlined customer
experience with direct on- and off-ramps between the traditional and
decentralized financial worlds. This means customers can use
self-custody wallets without having to give up convenience in exchange
for security or self-hosted options.

At scale, a competitive network of PFIs will also bring wallets more
liquidity and competition for their customers, which means lower fees
and faster transaction times.

The tbDEX protocol does not enforce any specific requirements upon
wallet implementations. Wallet developers may design features and
functionality that yield their desired user experience. For example, a
wallet could algorithmically select the PFI based on speed, cost, or
track record --- or delegate that choice to the owner of the wallet. A
wallet developer could choose to pre-select which PFIs a given offer
should be sent to --- or choose to request and verify the credentials of
various PFIs ahead of time by conducting discovery and evaluation prior
to the first offer. A wallet could also choose to leave selection of
PFIs entirely up to their customer. Generally speaking, we would
recommend the following:

• Portability. Individuals or organizations should be able to seamlessly
move all of their credentials to another wallet. The wallet should never
claim or assume any sense of ownership over an individual's VCs.

• Consent-Driven. Wallet implementations must always ask for the
individual's consent prior to presenting VCs to other parties, and may
lean on storing their preferences to improve user experience.

3.2.2 Custodial wallets

It is reasonable to expect in institutional use cases that wallets are
hosted by an organisation on behalf of a user (including private key
material): this does not mean that DIDs and VCs are not able to be used
to facilitate these use cases, but can still be used to discover and
interact with PFIs on behalf of their users.

\end{document}
